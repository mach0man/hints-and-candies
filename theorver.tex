\documentclass[a4paper,12pt]{report}
\usepackage[T2A]{fontenc}
\usepackage[utf8]{inputenc}
\usepackage{amssymb,amsfonts}
\usepackage[fleqn]{amsmath}
\usepackage[russian,english]{babel}
\usepackage{anyfontsize}
\usepackage{graphicx}
\usepackage{fancyhdr}
\setlength{\headheight}{15.2pt}
\renewcommand{\headrulewidth}{0pt}
\pagestyle{fancy}

\usepackage{geometry}
\geometry{left=2cm}
\geometry{right=1.5cm}
\geometry{top=1cm}
\geometry{bottom=2cm}

\DeclareMathOperator{\sinc}{sinc}
\DeclareMathOperator{\Tr}{Tr}   %trace
\DeclareMathOperator{\Dim}{dim}   %trace
%\DeclareMathOperator{\tg}{tg}

\newcommand{\abs}[1]{\left| #1 \right|} % for absolute value
\newcommand{\norm}[1]{\lVert #1 \rVert} % for norm ||f||
\newcommand{\qnorm}[1]{\lVert #1 \rVert ^2} % for norm quadrat ||f||^2
\newcommand{\avg}[1]{\left< #1 \right>} % for average
\let\underdot=\d % rename builtin command \d{} to \underdot{}
\renewcommand{\d}[2]{\frac{d #1}{d #2}} % for derivatives
\newcommand{\dd}[2]{\frac{d^2 #1}{d #2^2}} % for double derivatives
\newcommand{\pd}[2]{\frac{\partial #1}{\partial #2}} % for partial derivatives
\newcommand{\pdd}[2]{\frac{\partial^2 #1}{\partial #2^2}} % for double partial derivatives
\newcommand{\pdc}[3]{\left( \frac{\partial #1}{\partial #2} \right)_{#3}} % for thermodynamic partial derivatives
\newcommand{\ket}[1]{\left| #1 \right>} % for Dirac bras
\newcommand{\bra}[1]{\left< #1 \right|} % for Dirac kets
\newcommand{\braket}[2]{\left< #1 \vphantom{#2} \right|\left. #2 \vphantom{#1} \right>} % for Dirac brackets
\newcommand{\matrixel}[3]{\left< #1 \vphantom{#2#3} \right| #2 \left| #3 \vphantom{#1#2} \right>} % for Dirac matrix elements
\newcommand{\comml}[2]{\left[ \hat{#1}, #2 \right]} % [A,...]
\newcommand{\commr}[2]{\left[ #1, \hat{#2} \right]} % [...,B]
\newcommand{\comm}[2]{\left[ \hat{#1}, \hat{#2} \right]} % for commutator for only two operators like [A,B]
\newcommand{\grad}[1]{\gv{\nabla} #1} % for gradient
\let\divsymb=\div % rename builtin command \div to \divsymb
\renewcommand{\div}[1]{\gv{\nabla} \cdot #1} % for divergence
\newcommand{\curl}[1]{\gv{\nabla} \times #1} % for curl

\newcommand{\praiseme}[1]{\scalebox{.4}{ \textcircled{\scalebox{.5}{CC}} \textcircled{\scalebox{.5}{BY}} \textcircled{\scalebox{.5}{SA}} #1, \the\day.\the\month.\the\year}}

\begin{document}

\fontsize{20}{24}
\center{\bfseries Теория вероятности}
\line(1,0){500}
\fontsize{12}{15}
\begin{enumerate}
\item Правило произведения
\item Перестановки $ P_n = n! $
\item Размещения $ A^k_n = \frac{n!}{(n-k)!} $
\item Сочетания $ C^k_n = \frac{n!}{k!(n-k)!} $
\item Гипергеометрическое распределение: M из N всего объектов выделены. P(A:\{m выдел. среди n вытянутых\}):\\
      $ P(A)=\frac{C^m_M C^{n-m}_{N-M}}{C^n_N} $
\item Из N частиц: ${n_1,n_2,...,n_k}$ - k типов. Выбирается m, A:\{${m_1,m_2,...,m_k}$\}.\\
      $ P(A)=\frac{\prod\limits_k C^{m_k}_{n_k}}{C^m_N} $
\item Формула разбиения множества на группы: l различных групп n элементов всего:
      $ \left\{ \begin{smallmatrix}
      n_1 \rightarrow 1\, gr. \\n_2 \rightarrow 2\, gr. \\...\\n_l\rightarrow l\, gr.
      \end{smallmatrix} \right\} $\\
      $ P=\frac{k}{n} \Leftarrow k=\frac{n!}{\prod\limits_k n_k!} $
\end{enumerate}
\line(1,0){500}
\begin{enumerate}
\item $A \subset B$ - А влечёт В, A - подмножество В
\item $A\subset B,\,B\subset A \Rightarrow A=B$
\item $\bar{A}$ - противоположное событие
\item $\varnothing$ - невозможное событие
\item $\bar{\varnothing}=\Omega$
\item $A\cap B = A\cdot B$ - пересечение = ``и''
\item $A\cap B = \varnothing$ - несовместные
\item $A\cup B$ - объединение = ``или''
\item $A\cap B=\varnothing \Rightarrow A\cup B = A+B$
\item $A\backslash B = A\cap\bar{B}$ - только А без В
\item $A \circ B = A\vartriangle B = (A\cup B)\backslash(A\cap B)=(A\cup B)\cap\overline{(A\cap B)} $ - симметрическая разность
\item $\left\{\begin{array}{l}A\cap A=A\\ A\cup A=A\end{array}\right.$;
      $\left\{\begin{array}{l}A\cap \Omega=A\\ A\cup \Omega=\Omega\end{array}\right.$;
      $\left\{\begin{array}{l}A\cap \varnothing=\varnothing\\ A\cup \varnothing=A\end{array}\right.$;
      $ \Omega\backslash A=\bar{A} $
\item $\left\{\begin{array}{l} A\cup B = B\cup A\\A\cap B = B\cap A\end{array}\right.$;
      $\left\{\begin{array}{l} A\cup(B\cup C)=(A\cup B)\cup C \\ A\cap(B\cap C)=(A\cap B)\cap C \end{array}\right.$;
      $\left\{\begin{array}{l} A\cup(B\cap C)=(A\cup B)\cap(A\cup C)\\A\cap(B\cup C)=(A\cap B)\cup(A\cap C)\end{array}\right.$
\item $\left\{\begin{array}{l} \overline{A\cup B}=\bar{A}\cap\bar{B}\\\overline{A\cap B}=\bar{A}\cup\bar{B}\end{array}\right.$ - законы де Моргана
\item $A\subset B\Rightarrow\bar{B}\subset\bar{A}=\bar{A}\supset\bar{B}$
\end{enumerate}
\line(1,0){500}
\vfill \hfill \scalebox{.2}{ \textcircled{\scalebox{.5}{CC}} \textcircled{\scalebox{.5}{BY}} \textcircled{\scalebox{.5}{SA}} Xtotdam, \the\year}
\end{document}
