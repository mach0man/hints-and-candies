\documentclass[a4paper,12pt]{report}
\usepackage[T2A]{fontenc}
\usepackage[utf8]{inputenc}
\usepackage{amssymb,amsfonts}
\usepackage[fleqn]{amsmath}
\usepackage[russian,english]{babel}
\usepackage{anyfontsize}
\usepackage{graphicx}
\usepackage{fancyhdr}
\setlength{\headheight}{15.2pt}
\renewcommand{\headrulewidth}{0pt}
\pagestyle{fancy}

\usepackage{geometry}
\geometry{left=2cm}
\geometry{right=1.5cm}
\geometry{top=1cm}
\geometry{bottom=2cm}

\DeclareMathOperator{\sinc}{sinc}
\DeclareMathOperator{\Tr}{Tr}   %trace
\DeclareMathOperator{\Dim}{dim}   %trace
%\DeclareMathOperator{\tg}{tg}

\newcommand{\abs}[1]{\left| #1 \right|} % for absolute value
\newcommand{\norm}[1]{\lVert #1 \rVert} % for norm ||f||
\newcommand{\qnorm}[1]{\lVert #1 \rVert ^2} % for norm quadrat ||f||^2
\newcommand{\avg}[1]{\left< #1 \right>} % for average
\let\underdot=\d % rename builtin command \d{} to \underdot{}
\renewcommand{\d}[2]{\frac{d #1}{d #2}} % for derivatives
\newcommand{\dd}[2]{\frac{d^2 #1}{d #2^2}} % for double derivatives
\newcommand{\pd}[2]{\frac{\partial #1}{\partial #2}} % for partial derivatives
\newcommand{\pdd}[2]{\frac{\partial^2 #1}{\partial #2^2}} % for double partial derivatives
\newcommand{\pdc}[3]{\left( \frac{\partial #1}{\partial #2} \right)_{#3}} % for thermodynamic partial derivatives
\newcommand{\ket}[1]{\left| #1 \right>} % for Dirac bras
\newcommand{\bra}[1]{\left< #1 \right|} % for Dirac kets
\newcommand{\braket}[2]{\left< #1 \vphantom{#2} \right|\left. #2 \vphantom{#1} \right>} % for Dirac brackets
\newcommand{\matrixel}[3]{\left< #1 \vphantom{#2#3} \right| #2 \left| #3 \vphantom{#1#2} \right>} % for Dirac matrix elements
\newcommand{\comml}[2]{\left[ \hat{#1}, #2 \right]} % [A,...]
\newcommand{\commr}[2]{\left[ #1, \hat{#2} \right]} % [...,B]
\newcommand{\comm}[2]{\left[ \hat{#1}, \hat{#2} \right]} % for commutator for only two operators like [A,B]
\newcommand{\grad}[1]{\gv{\nabla} #1} % for gradient
\let\divsymb=\div % rename builtin command \div to \divsymb
\renewcommand{\div}[1]{\gv{\nabla} \cdot #1} % for divergence
\newcommand{\curl}[1]{\gv{\nabla} \times #1} % for curl

\newcommand{\praiseme}[1]{\scalebox{.4}{ \textcircled{\scalebox{.5}{CC}} \textcircled{\scalebox{.5}{BY}} \textcircled{\scalebox{.5}{SA}} #1, \the\day.\the\month.\the\year}}

%\renewcommand*{\familydefault}{\sfdefault}
\begin{document}
\rfoot{\praiseme{Xtotdam}}

\fontsize{20}{24}
\center{\bfseries OMM}
\line(1,0){500}
\fontsize{12}{15}
\begin{enumerate}[label=\textbf{\underline{\arabic*.}}]
\item \textbf{Перечислите основные этапы математического моделирования.}\\
      \begin{enumerate}
      \item Создание качественной модели (выясняются главные черты, особенности процесса)
      \item Создание математической модели (выделение существенных факторов и дополнительных условий)
      \item Изучение математической модели (обоснование $\rightarrow$ качественное исследование $\rightarrow$ численное исследование)
      \item Получение результатов, их интерпретация
      \item Использование полученных результатов
      \end{enumerate}
\item \textbf{Дайте определение детерминированной модели.}\\
      Если модель описывается некоторыми уравнениями, то она называется детерминированной.
\item \textbf{Дайте определение стохастической модели.}\\
      Если модель описывается вероятностными законами, то она называется стохастической.
\item \textbf{Что такое прямые задачи математического моделирования? Приведите примеры.}\\
      Прямая задача: все параметры исследуемой задачи известны и изучается поведение модели в различных условиях.\\
      Примеры: моделирование каскада химических реакций, молекулярная динамика.
\item \textbf{Что такое обратные задачи математического моделирования? Приведите примеры.}\\
      Обратные задачи - тип задач, часто возникающий во многих разделах науки, значения параметров модели должны быть получены из наблюдаемых данных. \Wiki\\
      Примеры: задачи распознавания (электроразведка, дефектоскопия), задачи синтеза.
\item \textbf{В чем состоит принцип аналогий в математической физике? Приведите примеры.}\\
	При моделировании сложных систем стоит использовать аналогии с уже изученными явлениями.
\item \textbf{Приведите примеры, демонстрирующие универсальность математических моделей.}\\
      Пример: процессы колебаний в объектах различной природы\\
      \begin{enumerate}
      \item Колебательный электрический контур, состоящий из конденсатора и катушки индуктивности.
      \item Малые колебания при взаимодействии двух биологических популяций
      \item Простейшая модель изменения зарплаты и занятости
      \end{enumerate}
      Универсальность математических моделей есть отражение принципа материального единства мира.
\item \textbf{Что такое иерархия моделей? Приведите примеры.}\\
      Принцип ``от простого к сложному'': построение цепочки (иерархии) все более полных моделей, каждая их которых обобщает предыдущую, включая её в качестве составного случая.\\
      Пример: Модель многоступенчатой ракеты
\item \textbf{Как ставится простейшая задача Гурса?}\\
      $\left\{\begin{array}{l}
            u_{xy}=f(x,y),\,x>0,\,y>0\\
            u(x,0)=\varphi(x),\,u(0,y)=\varphi(y)\\
            u(0,0)=\varphi_1(0)=\varphi_2(0)
      \end{array}\right. $\\$\Rightarrow
      u(x,y)=\varphi_1(x)+\varphi_2(y)-u(0,0)+ \int\limits_0^y \int\limits_0^x f(\xi,\eta)d\xi d\eta $\\
\item \textbf{Как ставится общая задача Гурса?}\\
      $\left\{\begin{array}{l}
            u_{xy}+a(x,y)u_x+b(x,y)u_y+c(x,y)u=f(x,y),\,x>0,\,y>0\\
            u(x,0)=\varphi(x),\,u(0,y)=\varphi(y)\\
            u(0,0)=\varphi_1(0)=\varphi_2(0)
      \end{array}\right. $\\$\Rightarrow
      u(x,y)=\underbrace{\int\limits_0^y \int\limits_0^x F(\xi,\eta)d\xi d\eta}_{A\left[u\right]} + \Phi(x,y);
      \left\{\begin{array}{l}
            F = f-a(x,y)u_x-b(x,y)u_y-c(x,y)u\\
            \Phi = \varphi_1(x)+\varphi_2(y)-u(0,0)
      \end{array}\right. $\\$
      \Rightarrow u=A[u]+\Phi\;(u_n=A[u_{n-1}]+\Phi,\,u_0=0) $
\item \textbf{Как ставится общая задача Коши в простейшем случае.}
\item \textbf{Поставьте общую задачу Коши.}\\
      $\left\{\begin{array}{l}
            u_{xy}=f(x,y),\,(x,y)\in D^+ \;\;(1)\\
            u(x,y)=\varphi(x,y),\,(x,y)\in C\\
            \pd{u}{n}(x,y)=\psi(x,y),\,(x,y)\in C
      \end{array}\right. $\\
      $\Rightarrow u(M)=\frac{\varphi(A)+\varphi(B)}{2}-\frac{1}{2}\int\limits_{AB}(u_y dy-u_x dx)+\int\limits_D f(x,y)dxdy$\\
\item \textbf{Какими свойствами должна обладать кривая С, на которой ставятся дополнительные условия в общей задаче Коши?}\\
      \begin{enumerate}
      \item C - не характеристика уравнения (1)
      \item Любая характеристика (1) пересекает С только один раз
      \end{enumerate}
\item \textbf{Дайте определение Функции Римана.}\\
      $\left\{\begin{array}{l}
            \mathcal{K}[v]=0 \text{ в }D\\
            v_x|_{y=y_0}=0 \Rightarrow v|_{y=y_0}=\const\\
            v_y|_{x=x_0}=0 \Rightarrow v|_{x=x_0}=\const\\
            v_{(x_0,y_0)}=1
      \end{array}\right.$\\
      $v$ - единственное решение этой задачи Гурса - функция Римана.
\item \textbf{Приведите простейший пример функции Римана.}\\
      $v(x,x_0,y,y_0)=J_0(\frac{c}{a}\sqrt{(x-x_0)(y-y_0)}) $ для задачи\\
      $\left\{\begin{array}{l}
            u_{tt}=a^2u_{zz}+c^2u+f(z,t),\,t>o,\,z\in(-\infty,\infty)\\
            u|_{t=0}=\varphi(z)\\
            u_t|_{t=0}=\psi(z)
      \end{array}\right.$
\item \textbf{Какие дифференциальные операторы называются сопряженными?}\\
      Два дифференциальных оператора - сопряженные, если разность $v\mathcal{L}[u]-u\mathcal{K}[v]$ является суммой частных производных по $x$ и $y$ от некоторых выражений $P$ и $Q$. Другими словами, если справедлива двумерная формула Грина\\
      $v\mathcal{L}[u]-u\mathcal{K}[v]=\frac1{2}(\pd{Q}{x}-\pd{P}{y}) $.
\item \textbf{Что произойдет, если характеристика уравнения общей задачи Коши пересечет кривую С, на которой заданы дополнительные условия, более чем в одной точке?}\\
      В таком случае потеряется произвольность $u(M_1)$, которая будет определяться как\\
      $ u(M_1)=\frac{u(A)V(A)+u(B_1)V(B_1)}{2}+\int\limits_{D_1}Vfdxdy-\frac{1}{2}\int\limits_{AB_1}Pdx+Qdy$\\
      $\left\{\begin{array}{l}
            V_{xy}=0,\,(x,y)\in D\\
            V_x|_{AM}=0,\,V_y|_{AM}=0,\,V(M)=1
      \end{array}\right.;\;
      \left\{\begin{array}{l}
            P[u,V]=V_x u-Vu_x\\
            Q[u,V]=V_y u-Vu_y
      \end{array}\right.$
\item \textbf{Как ставится задача Стефана?}\\
      $\left\{\begin{array}{l}
            \pd{u_1}{t}=a_1^2\pdd{u_1}{x},\,0<x,\xi\\
            \pd{u_2}{t}=a_2^2\pdd{u_2}{x},\,\xi<x<\infty
      \end{array}\right.$\\
      $\left\{\begin{array}{l}
            u_1 = T_1,\,x=0\\
            u_2 = T,\,t=0
      \end{array}\right.$\\
      $u_1=u_2=0,\,x=\xi$\\
      $ k_1\pd{u_1}{x}|_{x=\xi}-k_2\pd{u_2}{x}|_{x=\xi}=\lambda\rho\d{\xi}{t}$
\item \textbf{Какой физический смысл имеет задача Стефана?}\\
      Она описывает процесс фазового перехода, например кристаллизацию жидкости.
\item \textbf{В чем состоит метод подобия?}
\item \textbf{Как ставится задача сорбции?}\\
      $\left\{\begin{array}{l}
            -V\pd{u}{x}=\pd{u}{t}+\pd{a}{t},\,x>0,\,t>0\\
            \pd{a}{t}=\beta(u-\gamma a),\,x>0,\,t>0\;\text{(ур-е кинетики сорбции)}\\
            a(x,0)=0,\,x\geq 0;\;u(x,0)=0,\,x>0\\
            u(0,t)=u_0,\,t\geq 0
      \end{array}\right.$\\
      $V$ - скорость газа;\\
      $a(x,t)$ - количество газа, поглощенного единицей объема сорбента;\\
      $u(x,t)$ - концентрация газа, находящегося в порах сорбента в слое $x$;\\
      $\pd{a}{t}$ - расход газа на увеличение сорбированного количества газа;\\
      $\pd{u}{t}$ - расход газа на повышение свободной концентрации в порах сорбента;\\
      $u_0$ - концентрация газа на входе;\\
      $\beta$ - кинетический коэффицент;\\
      $y$ - концентрация газа, находящегося в равновесии с сорбированным количеством газа.
\item \textbf{Напишите уравнение кинетики сорбции.}\\
      $\pd{a}{t}=\beta(u-\gamma a),\,x>0,\,t>0 $
\item \textbf{Что такое изотерма сорбции? Приведите примеры.}\\
      Изотерма сорбции - зависимость концентрации вещества в неподвижной фазе от его концентрации в подвижной при постоянной температуре $ \equiv a = f(y)|_{T=\const} $. Угол наклона изотермы сорбции определяет коэффициент распределения вещества между фазами. (Wiki)\\
      И. Ленгмюра $a=\frac{yu_0}{\gamma(u_0+py)} \xrightarrow[p\to 0]{}
      a=\underbrace{\frac1{\gamma}}_\text{к-та Генри}y$ - и. Генри
\item \textbf{Рассмотрите поведение на бесконечности решения уравнения Гельмгольца при различных видах коэффициента С.}\\
      $\Delta u+cu=-f$\\
      \begin{enumerate}
      \item $c = -\kappa^2 < 0$\\
            $ u^\pm(M) = \frac{1}{4\pi}\int\limits_D\frac{e^{\pm\kappa r_{QM}}}{r_{QM}}f(Q)dV_Q $,
            $f(M)$ - финитна, $ \operatorname{supp}f \subset D $\\
      \item $c=k^2,\,k=\bar{k}+i\bbar{k},\,\bbar{k}>0 $\\
            $\exists !$ решение, $\xrightarrow[\infty]{}0\,:\;
            u(M) = \frac{1}{4\pi}\int\limits_D\frac{e^{ikr_{QM}}}{r_{QM}}f(Q)dV_Q $ -
            при временной зависимости $e^{i\omega t}$ - это расходящаяся волна
      \item $c=k^2>0 $\\
            $u^\pm(M) = \frac{1}{4\pi}\int\limits_D\frac{e^{\pm kr_{QM}}}{r_{QM}}f(Q)dV_Q $ -
            оба одинаково убывают на $\infty$
      \end{enumerate}
\item \textbf{Сформулируйте для неограниченной области теорему единственности решения уравнения Гельмгольца в случае отрицательного коэффициента С.}\\
      \Th Классическое решение уравнения $\Delta u -\kappa^2 u = -f(M)$, равномерно стремящееся к нулю на бесконечности - единственно.
\item \textbf{Напишите условие излучения Зоммерфельда в трехмерном случае.}\\
      $\left\{\begin{array}{l}
            u(M)=O(\frac{1}{r})\\
            \pd{u}{r}-iku=o(\frac{1}{r})
      \end{array}\right.$
\item \textbf{Напишите условия излучения Зоммерфельда в двумерном случае.}\\
      $\left\{\begin{array}{l}
            u(M)=O(\frac{1}{\sqrt{r}})\\
            \lim\limits_{r\to\infty}(\pd{u}{r}-iku)=0
      \end{array}\right.$
\item \textbf{В каком случае и для чего ставятся условия излучения Зоммерфельда?}
      \textcolor{red}{?}  Условия ставятся для решений уравнения Гельмгольца.
\item \textbf{Сформулируйте принцип предельного поглощения.}
\item \textbf{Сформулируйте принцип предельной амплитуды.}
\item \textbf{Приведите пример постановки парциальных условий излучения.}
\item \textbf{Какой излучатель называется квадрупольным?}
\item \textbf{Как ставится задача математической теории дифракции?}\\
      $p_i=\const,\,\rho_i=\const,\;i=\overline{0,n}$\\
      $\left\{\begin{array}{l}
            \Delta V_i+k_i^2 V_i=-f_i,\,M\in D_i,\,i=\overline{0,n}\\
            V_i = V_0,\,M\in S_i\\
            p_i \pd{V_i}{n}=p_0\pd{V_0}{n},\,M\in S_i,\,i=\overline{1,n}\\
            V_0(M) = O(\frac{1}{r})\\
            \pd{V_0}{r}-ik_0 V_0 = o(\frac{1}{r})\\
            k_i^2=\frac{\rho_i}{p_i}\omega^2,\,i=\overline{0,n}
      \end{array}\right.$
\item \textbf{Что такое автомодельное решение?}\\
      Автомодельными решениями квазилинейного уравнения теплопроводности мы будем называть такие его частные решения специального вида, которые могут быть получены путем интегрирования некоторых обыкновенных дифференциальных уравнений, аргументы искомых функций которых представляют собой комбинацию независимых переменных $x$ и $t$.
\item \textbf{Дайте определение квазилинейного уравнения теплопроводности.}
\item \textbf{Сформулируйте основные свойства квазилинейного уравнения теплопроводности.}
\item \textbf{Что такое тепловые волны? При каких условиях они возникают?}
\item \textbf{Что такое режимы с обострением? Приведите примеры.}
\item \textbf{При каком режиме с обострением образуется стоячая тепловая волна?}
\item \textbf{Напишите квазилинейное уравнение переноса.}
\item \textbf{Напишите уравнение характеристик для квазилинейного уравнения переноса.}
\item \textbf{Могут ли характеристики квазилинейного уравнения переноса пересекаться? Что это означает физически?}
\item \textbf{В чем состоит явление опрокидывания волн? Как его можно объяснить?}
\item \textbf{В каких случаях необходимо строить обобщенное решение квазилинейного уравнения переноса?}
\item \textbf{Напишите условие на разрыве (условие Гюгонио-Ренкина).}
\item \textbf{Напишите уравнение Кортевега – де Фриза.}\\
      $\eta_t+c_0(1+\frac3{2h_0}\eta)\eta_x+\frac{h_0^2}{6}c_0\eta_{xxx}=0 $,\\
      $h_0$ - глубина жидкости;\\
      $c_0 = \sqrt{gh_0}$ - скорость длинных волн на мелкой воде.\\
      $u_t-6uu_x+u_{xxx}=0 $ - канонический вид
\item \textbf{Для решения какой нелинейной задачи применяется схема решения обратной задачи рассеяния?}
\item \textbf{Изложите схему решения обратной задачи рассеяния.}\\
\line(1,0){500}
\item \textbf{Что такое солитонные решения?}
\item \textbf{Решением какого уравнения являются солитоны?}
%Lecture 9
\item \textbf{В чем состоит принцип сведения краевых задач к вариационным задачам (принцип Дирихле)?}
\item \textbf{Как ставится вариационная задача на собственные значения?}
\item \textbf{Что такое вариационные и что такое проекционные алгоритмы? Приведите примеры.}
\item \textbf{В чем состоит метод Ритца?}
\item \textbf{Что такое энергетическое пространство? В каком случае его можно построить?}
\item \textbf{Какие краевые условия называются главными, и какие естественными?}
\item \textbf{В каких случаях метод Ритца неприменим?}
\item \textbf{В чем состоит метод Галеркина?}
\item \textbf{В чем состоит обобщенный метод моментов?}
\item \textbf{В чем состоит метод наименьших квадратов?}
\item \textbf{Дайте определение разностной схемы.}
\item \textbf{Что такое условие согласования норм?}
\item \textbf{Дайте определение аппроксимации разностной задачей исходной дифференциальной задачи.}
\item \textbf{Дайте определение устойчивости разностной схемы.}
\item \textbf{Дайте определение сходимости разностной схемы.}
\item \textbf{Что означает, что разностная задача имеет m-й порядок точности?}
\item \textbf{Дайте определение корректной постановки разностной схемы.}
\item \textbf{Что означает выражение: из аппроксимации и устойчивости разностной схемы следует ее сходимость? Для каких разностных схем оно справедливо?}
\item \textbf{Что такое шаблон разностного оператора? Приведите примеры.}
\item \textbf{Приведите пример явной разностной схемы. В чем ее достоинства и недостатки?}
\item \textbf{Приведите пример неявной разностной схемы. В чем ее достоинства и недостатки?}
\item \textbf{Напишите условия устойчивости явной разностной схемы.}
\item \textbf{Приведите пример безусловно устойчивой схемы.}
\item \textbf{Приведите пример экономичной разностнойсхемы.}
\item \textbf{Напишите схему переменных направлений (схему Письмена-Рэкфорда).}
\item \textbf{Дайте определение однородной разностной схемы.}
\item \textbf{Что такое шаблонные функционалы?}
\item \textbf{Дайте определение консервативной разностной схемы.}
\item \textbf{Приведите пример консервативной разностной схемы.}
\item \textbf{Приведите пример неконсервативной разностной схемы.}
\item \textbf{Какие методы построения консервативной разностных схем вам известны?}
\item \textbf{В чем состоит интегро-интерполяционный метод (метод баланса)?}
\item \textbf{Опишите алгоритм метода конечных элементов.}
\item \textbf{Приведите пример простейшего базиса метода конечных элементов.}
\item \textbf{Сформулируйте необходимое спектральное условие устойчивости Неймана для решения разностной задачи Коши.}
\item \textbf{Что такое асимптотическая формула?}
\item \textbf{Какие члены асимптотической формулы называются остаточными?}
\item \textbf{Может ли асимптотический ряд быть расходящимся?}
\item \textbf{Может ли асимптотическая формула обеспечить произвольную степень точности? Если да, то приведите пример.}
\item \textbf{Что в асимптотических методах понимается под возмущением?}
\item \textbf{Что такое регулярное возмущение?}
\item \textbf{Что такое сингулярное возмущение?}
\item \textbf{Какое решение невозмущенного уравнения называется устойчивым?}
\item \textbf{Что такое область влияния (притяжения) корня невозмущенного уравнения?}
\item \textbf{В чем состоит метод ВКБ?}
\item \textbf{Опишите алгоритм метода Крылова-Боголюбова. Для решения каких задач он применяется?}
\item \textbf{Почему метод Крылова-Боголюбова называется методом усреднения?}
\item \textbf{Что такое аттрактор? Что такое странный аттрактор?}
\item \textbf{Дайте определение фрактала.}
\item \textbf{Какие фракталы называются конструктивными? Приведите примеры.}
\item \textbf{Какие фракталы называются динамическими? Приведите примеры.}
\item \textbf{Приведите примеры расчета размерности конструктивных фракталов.}
\item \textbf{Что такое дендриты? Приведите примеры.}
\item \textbf{Что такое вейвлет-анализ? Для чего он применяется?}
\item \textbf{Почему функции Хаара, функции Литлвуда-Пелли и функции Габора не используются в качестве базисных функций в вейвлет-анализе?}
\item \textbf{Что такое материнский (анализирующий) вейвлет?}
\item \textbf{Перечислите основные свойства функций вейвлет-семейства.}
\item \textbf{В чем состоит преимущество вейвлет-преобразования перед фурье-преобразованием?}
\item \textbf{Приведите примеры применения вейвлет-анализа.}
\item \textbf{Что такое диссипативные структуры?}
\item \textbf{Что изучает синергетика?}
\item \textbf{Опишите модель брюсселятора.}
\item \textbf{Что такое термодинамическая ветвь?}
\item \textbf{Перечислите основные свойства систем, в которых возможны явления самоорганизации и возникновения структур.}
\end{enumerate}
\vfill\line(1,0){500}
\end{document}
